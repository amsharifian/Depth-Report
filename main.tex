%   DOCUMENT CLASS  %%%%%%%%%%%%%%%%%%%%%%%%%%%%%%%%%%%%%%%%%%%%%%%%%%%%%%%%%%%
%
%   Use the `sfuthesis` class to format your thesis. If your program does not
%   require a thesis defence, use the class option `undefended` like so:
%
%     \documentclass[undefended]{sfuthesis}
%
%   To generate a signature page for your defence, use the `sfuapproval` class
%   instead, by replacing the below line with
%
%     \documentclass{sfuapproval}
%
%   For more information about thesis formatting requirements, go to
%
%     http://www.lib.sfu.ca/help/publish/thesis
%
%   or ask a thesis advisor at the SFU Research Commons.
%

\documentclass{sfuthesis}



%   DOCUMENT METADATA  %%%%%%%%%%%%%%%%%%%%%%%%%%%%%%%%%%%%%%%%%%%%%%%%%%%%%%%%
%
%   Fill in the following information for the title page and approval page.
%

\title{An example of a thesis or dissertation on the subject of your degree}
\thesistype{Dissertation}
\author{Amirali Sharifian}
\previousdegrees{%
	M.Sc., Simon Fraser University, 2016\\
	B.Sc., Isfahan University of Technology, 2014}
\degree{Doctor of Philosophy}
\discipline{Computer Science}
\department{Department of Computing Science}
\faculty{Faculty of Mad Science}
\copyrightyear{2019}
\semester{Summer 2019}
\date{January 10, 2017}

\keywords{HLS, Accelerator design, Synthesis, Configurable architecture, FPGA }

\committee{%
	\chair{Pamela Isely}{Professor}
	\member{Emmett Brown}{Senior Supervisor\\Professor}
	\member{Bonnibel Bubblegum}{Supervisor\\Associate Professor}
	\member{James Moriarty}{Supervisor\\Adjunct Professor}
	\member{Kaylee Frye}{Internal Examiner\\Assistant Professor\\School of Engineering Science}
	\member{Hubert J.\ Farnsworth}{External Examiner\\Professor\\Department of Quantum Fields\\Mars University}
}



%   PACKAGES %%%%%%%%%%%%%%%%%%%%%%%%%%%%%%%%%%%%%%%%%%%%%%%%%%%%%%%%%%%%%%%%%%
%
%   Add any packages you need for your thesis here.
%   You don't need to call the following packages, which are already called in
%   the sfuthesis class file:
%
%   - appendix
%   - etoolbox
%   - fontenc
%   - geometry
%   - lmodern
%   - nowidow
%   - setspace
%   - tocloft
%
%   If you call one of the above packages (or one of their dependencies) with
%   options, you may get a "Option clash" LaTeX error. If you get this error,
%   you can fix it by removing your copy of \usepackage and passing the options
%   you need by adding
%
%       \PassOptionsToPackage{<options>}{<package>}
%
%   before \documentclass{sfuthesis}.
%
%   The following packages are a few suggestions you might find useful.
%
%   (1) amsmath and amssymb are essential if you have math in your thesis;
%       they provide useful commands like ``blackboard bold'' symbols and
%       environments for aligning equations.
%   (2) amsthm includes allows you to easily change the style and numbering of
%       theorems. It also provides an environment for proofs.
%   (3) graphicx allows you to add images with \includegraphics{filename}.
%   (4) hyperref turns your citations and cross-references into clickable
%       links, and adds metadata to the compiled PDF.
%   (5) pdfpages lets you import pages of external PDFs using the command
%       \includepdf{filename}. You will need to do this if your research
%       requires an Ethics Statement.
%

\usepackage{amsmath}                            % (1)
\usepackage{amssymb}                            % (1)
\usepackage{amsthm}                             % (2)
\usepackage{graphicx}                           % (3)
\usepackage[pdfborder={0 0 0}]{hyperref}        % (4)
\usepackage{minted}
% \usepackage[outputdir=cache]{minted}
\usepackage{filecontents}
% \usepackage{natbib}
\usepackage{bibentry}
\usepackage{xcolor}

\nobibliography*

\definecolor{LightGray}{gray}{0.9}
%\definecolor{DarkGray}{gray}{0.1}

%\pagecolor{DarkGray}

% \usemintedstyle{borland}

%New colors defined below
\definecolor{codegreen}{rgb}{0,0.6,0}
\definecolor{codegray}{rgb}{0.5,0.5,0.5}
\definecolor{codepurple}{rgb}{0.58,0,0.82}
\definecolor{backcolour}{rgb}{0.95,0.95,0.92}

% \usepackage{pdfpages}                         % (5)
% ...
% ...
% ...
% ... add your own packages here!




%   OTHER CUSTOMIZATIONS %%%%%%%%%%%%%%%%%%%%%%%%%%%%%%%%%%%%%%%%%%%%%%%%%%%%%%
%
%   Add any packages you need for your thesis here. We've started you off with
%   a few suggestions.
%
%   (1) Use a single word space between sentences. If you disable this, you
%       will have to manually control spacing around abbreviations.
%   (2) Correct the capitalization of "Chapter" and "Section" if you use the
%       \autoref macro from the `hyperref` package.
%   (3) The LaTeX thesis template defaults to one-and-a-half line spacing. If
%       your supervisor prefers double-spacing, you can redefine the
%       \defaultspacing command.
%

\frenchspacing                                    % (1)
\renewcommand*{\chapterautorefname}{Chapter}      % (2)
\renewcommand*{\sectionautorefname}{Section}      % (2)
\renewcommand*{\subsectionautorefname}{Section}   % (2)
% \renewcommand{\defaultspacing}{\doublespacing}  % (3)
% ...
% ...
% ...
% ... add your own customizations here!


%% Include command tex file
\include{commands}


%   FRONTMATTER  %%%%%%%%%%%%%%%%%%%%%%%%%%%%%%%%%%%%%%%%%%%%%%%%%%%%%%%%%%%%%%
%
%   Title page, committee page, copyright declaration, abstract,
%   dedication, acknowledgements, table of contents, etc.
%
%   If your research requires an Ethics Statement, download one from the
%   SFU library website and uncomment the appropriate lines below.
%


% extensible architecture

\begin{document}

\frontmatter
\maketitle{}
% \makecommittee{}

%\addtoToC{Ethics Statement}%
%\includepdf[pagecommand={\thispagestyle{plain}}]{ethicsstatement.pdf}%
%\clearpage

\begin{abstract}
In recent years, the computing landscape has seen an increasing shift towards specialized accelerators.
Reconfigurable architectures like FPGAs are particularly promising for accelerator implementation as they can offer performance and energy efficiency improvements over CPUs and GPUs and are more flexible than fixed-function ASICs.
Unfortunately, adoption of reconfigurable hardware has been limited by their associated tools and programming models:
hardware description languages (HDLs) lack abstractions for productivity and are difficult to target directly from higher level languages.
The alternative to HDLs to support higher level of abstraction is High-level synthesis (HLS) tools.

HLS techniques have been proposed to improve the productivity of hardware designers by automatically generating the hardware from a high-level description of an algorithm.
In HLS tools, front-end captures system behavior with a model of computation in a standard language such as C, C++, SystemC as an input.
The language components, in this case, are in the form of untimed mathematical expressions and nested, pipeline and parallel loops.
And then, the compiler schedules the input algorithm and applies optimizations such as inner loop pipelining, unrolling, memory banking and buffering to generate RTL.

The purpose of this work is first to have a clear definition of hardware accelerators and different level of abstraction to express a design. Then we look at state-of-the-art HLS approaches to automatically synthesis hardware from the higher-level description of an algorithm.
Finally, we look at the limitation of each technique and outline what we believe is the main challenge with the current HLS techniques.

\end{abstract}


% \begin{dedication}
% 	This is an optional page.
% \end{dedication}


% \begin{acknowledgements}
% 	This is an optional page.
% \end{acknowledgements}

%%%%% ENABLE THIS PART LATTER %%%%%%

\addtoToC{Table of Contents}%
\tableofcontents%
\clearpage

% \addtoToC{List of Tables}%
% \listoftables%
% \clearpage

% \addtoToC{List of Figures}%
% \listoffigures%
% \clearpage





%   MAIN MATTER  %%%%%%%%%%%%%%%%%%%%%%%%%%%%%%%%%%%%%%%%%%%%%%%%%%%%%%%%%%%%%%
%
%   Start writing your thesis --- or start \include ing chapters --- here.
%

\mainmatter%

%!TEX root = main.tex

\chapter{Background}
\label{chapter:background}

\section{Why application specific accelerators?}
\label{background:accel}


Recent trends in technology scaling, the availability of large amounts of data, and novel algorithmic breakthroughs have spurred accelerator architecture research.
In a general-purpose microprocessor, the overhead of instruction processing is much higher than the actual operations performed by each instruction.
This overhead includes the necessary steps to fetch and decode the instructions, provide required operands for the instructions, and perform the necessary bookkeeping to ensure correctness when multiple instructions are executing in the microprocessor.
Conversely, application specific hardware are faster and lower in power consumption than general-purpose processors because they eliminate most of the overhead of a general purpose processor~\cite{chung_micro_2010, hameed_asplos_2010_understanding}.
Although fixed-function accelerators are more energy efficient than software running a general-purpose processor, they are not a suitable solution for applications that change frequently.
As an alternative to fixed-function accelerators, reconfigurable architectures like field-programmable gate arrays (FPGAs) and coarse-grain reconfigurable architectures (CGRAs) have received renewed interest from academic researchers and industry practitioners alike, primarily due to their potential performance and energy efficiency benefits over conventional CPUs.
For instance, FPGAs are now being used to accelerate web search in datacenters at Microsoft and Baidu~\cite{catapult, baidu}, Amazon now offers FPGA instances as part of AWS~\cite{awsf1}, and Intel has announced products like in-package Xeon-FPGA systems~\cite{harp} and FPGA-accelerated storage systems~\cite{nand_flash}. Similarly, several recent research prototypes~\cite{dyser, triggered_instruction, scaledeep, scnn, plasticine, cgra_me} have explored various kinds of CGRAs at different granularity. Growing use of such reconfigurable architectures has made them more available to programmers now than ever before.
Although the flexibility of reconfigurable architectures enables changing the application by reconfiguring the accelerator, their programmability is still a major obstacle for their wide spread use.

\section{System-Design Challenges}


Reconfigurable devices, usually, accelerates part of the application which contains regular control flow and abundant data parallelism to achieve high performance and efficiency~\cite{spatial_computation, trips, govindaraju_hpca_2011}.
They can exploit: 1)multiple levels of nested parallelism, 2)data locality with custom data pipelines and 3)defining custom memory hierarchies.

Unfortunately, \textit{all the features that make reconfigurable architectures efficient also make them much more complex to program.}
For instance, in FPGAs, an accelerator design must account for the timing between pipelined signals and the physically limited compute and memory resources available on the target device.
It must also manage partitioning of data between local scratchpads and off-chip memory to achieve good data locality~\cite{gzip_2013_fpga}.


The combination of these complexities and market pressures, not the least
of which is reliability, we are finding that traditional design methods, in which
systems are designed directly at the low hardware or software levels, are fast
becoming infeasible~\cite{cascaval_taxonomy_accelerator}. This leads us to the well-known productivity gap generated by the disparity between the rapid paces at which design complexity has increased in comparison to that of design productivity~\cite{itrs}.
Figure~\ref{fig:productivity} shows the growth of design complexity and designer productivity
over time, as estimated by the Sematech in the mid-1990s. Design complexity
is fundamentally estimated by Moore’s Law, which predicts a 58%
annual increase in the number of transistors per chip. Sematech estimates that
designer productivity has grown and will continue to grow by only 21% per
year. The result is a wide and growing gap between the chips we can manufacture
and the chips we can design. 

\begin{figure}[h]
    \centering
    \includegraphics[width=\textwidth]{plots/productivity.pdf}
    \caption{Design complexity and designer productivity trends}
    \label{fig:productivity}
\end{figure}

One of the commonly-accepted solutions for closing the productivity gap as proposed by semiconductor roadmaps is to \emph{raise the level of abstraction in the design process.}
In order to achieve the acceptable productivity gains and to bridge the semantic gap between higher abstraction levels and low-level implementations, the goal now is to \emph{automate} the system-design process as much as possible.
However, to make automation possible we need to have: 1)A well-defined system abstraction level, well-known components of particular abstraction level and having a clear semantics for system-design languages.
The code, however, can be generated in a specific target language such as C>
In order to understand system-level possibilities more fully, we first explain the different abstraction levels involved in system design.

%!TEX root = ../main.tex

% NOTES:
%   * There is a shift to accelerator design
%   * CPUs are inefficient
%   * Fixed accelerators are hard to program
%   * Alternative is FPGA/CGRA
%   * There are challenges with programming FPGA/CGRA
%   * HLS is one solution

%


\section{Abstraction Levels}

The growing capabilities of silicon technology over the previous decades has forced design methodologies and tools to move to higher levels of abstraction.
In order to explain the relationship between different design methodologies on different abstraction levels, we will use the Y-Chart~\cite{walker_1985_y_model}~\ref{fig:y-chart}.

\begin{figure}[h]
    \centering
    \includegraphics[width=0.5\textwidth]{figures/Introduction/Y-Chart.pdf}
    \caption{Y-Chart}
    \label{fig:y-chart}
\end{figure}

Y-Chart divides design representation into three domains, three axises on the chart:
1)The behavioral domain describes the behavior, or functionality, of the design, ideally without any reference to the way this behavior is achieved by an implementation.
Behavioral represents a design as a black-box but it describes the output base on inputs over time.
What behavior representation doesn't specify is how the black-box is structured or how to build the black-box.
2)The structural domain describes the abstract implementation, or logical structure, of the design as a hierarchy of components and their interconnections.
In this specification, black-box is represented as a set of components and connections.
While, it's possible to drive the behavior of black-box from its components and its connections but understanding the behavior can be very difficult since it is obscured by the details of each component and connection.
3)Physical, usually called layout or board design.
Physical design describe different dimension of each component.

The Y-chart contains three axises to represent design domains.
It also contains concentric circles to defines multiple level of abstractions for a design.
Gajski~\cite{gajski_1992_high} proposed four levels of abstraction: system, processor, logic and circuit levels.
In this model, the name of each abstraction level is derived from the types of the components generated on that abstraction level.
For instance, on the processor level, we generate general purpose and \emph{custom} processors, or \emph{special-hardware components} such as memory controllers, arbiters, bridges and various interface components.
At the higher level, system level, we design standard or embedded systems consisting of processors, memories, buses, and other processor components.
In the next two other level, we describe the circuit in terms of registers and the data transfer between the registers, logic level. Finally, circuit level or transistor level, implements the behavior of logic gates.
In the rest of this report we focus on only the first two level of abstraction, processor and system.

On each abstraction level, we need a library of components to be used in building \emph{the structure} for a given \emph{behavior}.
The process of converting the given behavior into a structure for a given haviour is called \textbf{synthesis}.
Once a structure is defined and verified, we can proceed to the next lower abstraction level by further synthesizing each components to their corresponding structure.
Thus, each component in the library may have up to three different models representing three different axes in the Y-Chart: behavior or function; structure, which contains the components from the lower level of abstraction and the physical.
Fortunately, to design a new hardware all three models for each component are not typically needed most of the time.
Most of the methodologies presently in use, which we are going to discuss in the next chapter, perform design or synthesis on the system and processor levels.
At these two levels, every system component except memories is synthesized to the logic level, before the physical design is performed on the logic level.
Once the design is represented in terms of logic gates, depending on the target backend, we can perform layout placement, routing and used standard cells for each logic components, like wires and registers.
On the other hand, some components on the processor-and-system levels may be obtained as IPs and not synthesized, for example DSP blocks in FPGAs.
Therefore, their structure and physical design are known, on the level higher than logic level.
In that case, the physical design then may contain components of different sizes and from different levels of abstraction.
In the rest of this report we focus on the first two level of abstractions: \textit{Processor} and \textit{System}, how these abstraction can be expressed a programming languages and the process of \emph{synthesis} at these two levels.


\section{Behavioral Model}
\label{sec:processor_level_behavioral_model}

On each abstraction level we design components with different granularity with programming languages. At the processor level, the core building block of design is processing elements (PEs).
PEs are computational components that we define and design to do operations.
Each PE can be either a dedicated or a custom component that computes some specific functions.
It can be a general, like ALU at processors, or standard PE that can compute any function specified in some standard programming language.
The functionality or behavior of each PE can be specified in several different ways.
In the basic implementation of PEs, their functionality can be specified with mathematical expressions or formulas.
However, there is no limitation on what the functionally of a PE can be.
Therefore, we can expand the functionality of a PE and specify it with an algorithm in some \textit{programming language}.
For instance, we can define three PEs with three different mathematical expression: $y = |b|$, $x = |a|$ and $z = max (x, y)$.
Using only mathematical expressions and PEs, however, to define different type of computation is not sufficient.
To go beyond a mathematical expression and add the notion of control, one possible solution is to use Finite State Machines (FSMs).
A FSM is defined with a set of states and a set of transitions from state to state, which are taken when some input variables reach the required value.
Furthermore, each FSM generates some values for output variables in each state or during each transition. A FSM model can be made clock-accurate if each state is considered to take one clock cycle.
Figure~\ref{fig:fsm_model} shows an example of a FSM which connected our three PEs.
The example FSM has three states, S1, S2, and S3, and arcs representing state changes under different inputs.
Each state executes a computation represented by one or more arithmetic expressions or programming statements.
For example, in state S1, the FSM in Figure~\ref{fig:fsm_model} computes two functions, $x = |a|$ and $y = |b|$, and in state S3 it computes the function $z = max (x, y)$.
Hardware Definition Languages (HDLs) like Verilog and VHDL have support for both mathematical expression to become PEs and FSM model to support control structures at their behavioral level.
But there is a limitation in FSM model that makes it insufficient to express different type of computations.
FSM model is usually not clock-accurate since computation in each state may take more than one clock cycle.
Hence, for representing computation expressed by programming languages like C, FSM model is not adequate and we need a more comprehensive model.


\begin{figure}[h]
    \centering
    \includegraphics[width=0.4\textwidth]{figures/Introduction/FSM.pdf}
    \caption{FSM model}
    \label{fig:fsm_model}
\end{figure}


In general, programming languages consist of three elements: \code{if} statements, loops, and expressions.
An \code{if} statement has two parts, then and else, in which then is executed if the conditional expression given in the \code{if} statement is true, otherwise the \code{else} part is executed.
In each of the then or else parts, the \code{if} statement computes \emph{a set of expressions called a Basic Block (BB)}.
The \code{if} statement can also be used in the loop construct to represent loop iterations, which are executed as long as the condition in the \code{if} statement is true.
As a result, the combination of \code{if} statements, Control-Flow Graph (CFG) and BBs, Data-Flow Graph (DFG) can represent any programming-language code.
 
Figure~\ref{fig:c_example} shows an example of such a CFG and DFG combination. 
In this this example, we represent a loop with an \code{if} statement inside the loop iteration.
In each iteration, the loop construct executes BB1 and BB2 or BB3 depending on the value of \code{if} statement. At the end, the loop is exited if all all iterations are executed.

A Control-Data dependency graph shows explicitly the control dependencies between loop statements, if statements, and BBs, as well as the data dependences among operations inside a BB.
On can convert Control-Data dependency graph to a FSM by assigning a state to each BB and one state for the computation of each if conditional.
However, each state in such a FSM may need several clock cycles to execute its assigned BB or if condition.
Designing such FSM is challenging, we explain more in chapter 2.

\begin{listing}[!h]
    \begin{minipage}{0.5\textwidth}
        \begin{minted}{C++}
            for(int i = 0; i < N; i++){
                if(i % 2 == 0){
                    a = b + c;
                    b = c * 4;
                }
                else {
                    a = b - c;
                    c = b * 4;
                }
            }
        \end{minted}
    \end{minipage}
    \begin{minipage}{0.5\textwidth}
       \includegraphics[width=0.85\textwidth]{figures/Introduction/CFG.pdf}
    \end{minipage}
    \caption{A C program example}
    \label{fig:c_example}
\end{listing}


While Control-Data dependency graph can be used to for specifying a single processor, but it is not suffice for describing a complete system that consist of many communicating processors.
A system-level model must represent multiple process running in parallel or in pipelined mode, with introducing notion of concurrency and pipelining.
Moreover, we need to have notion of synchronization since each processor can run concurrently and they may exchange data between each other.
Language extensions such as Cilk~\cite{cilk}/OpenMP and languages like Go with the notion of communication channels provide such mechanism to specify system level behavior at the software.

\section{Structural Model}

As we discussed a processor’s behavioral model, whether defined by a program in C, Control-Data dependency graph or FSM, \emph{can be implemented with a set of register-transfer components}.
In such a structural model usually there are a controller and a datapath like general purpose processors.
A datapath consists of three main component. 1) a set of storage elements such as register files, scratchpads, and memories, 2) a set of functional units such as ALUs, multipliers, shifters, and other custom functional units, and 3)a set of busses.
All of these components may be allocated in different quantities, size and types and connected arbitrarily through busses.
Each component may take one or more clock cycles to execute, each component may be pipelined, and each component may have input or output latches.
In addition, in many processor designs, the entire datapath is pipelined in several stages in addition to components being pipelined by themselves.
The choice of components and datapath structure depends on the metrics such as performance or energy to be optimized for a particular implementation.

The datapath does not work stand alone without a controller.
The controller defines the state of the processor and issues the control signals for the datapath at each cycle.
The structure of the controller and its implementation varies depending on whether the processor is a standard processor, like RISC processors, a custom-design processor like special image processing processors. 
Figure~\ref{fig:proc_structure} shows an example of such control and datapath for a simple 2 stage RISCV sodor core\footnote{\url{https://github.com/librecores/riscv-sodor/wiki}}.
In the case of a standard processor, the controller is programmable with a program counter (PC), and an address generator (AG) that defines the next address to be loaded into the PC.
In the case of specific custom processors, the controller can be implemented with programmability concepts typical of standard processors.



\begin{figure}[!t]
    \centering
    \includegraphics[width=\textwidth]{figures/Introduction/2stage.pdf}
    \caption{Processor structure model -- 2 Stage RISCV core}
    \label{fig:proc_structure}
\end{figure}


\section{Processor-Level Synthesis}

Synthesis of standard processors starts with the instruction set (IS) of the processor.
In order to achieve the highest processor performance this process is done \textbf{manually} since standard processors try to achieve the highest performance and minimal power consumption at minimal cost.
Another reason for synthesizing processors manually is to minimize the design size and therefore fabrication cost for high-volume production, optimize the design as much as possible.
In contrast, the design or synthesis of a custom processor or a custom IP starts with the C code of an algorithm, which is usually converted to the corresponding Control-Data dependency graph, Figure~\ref{fig:c_example}, or FSM model, Figure~\ref{fig:fsm_model}, before synthesis and ends up with a custom processor containing the number and type of components connected as required by the given behavioral model.
This generation is usually called high-level synthesis or occasionally just processor synthesis.
Selecting components and the structure of a PE and \emph{defining register-transfer operations performed in each clock cycle is the task of processor-level synthesis.}

\begin{figure}[h]
    \centering
    \includegraphics[width=0.9\textwidth]{figures/Introduction/processor-synthesis.pdf}
    \caption{High-Level Synthesis (HLS)}
    \label{fig:proc_synthesis}
\end{figure}

In the process of synthesis there are different tasks that needs to be done to move from behavioral description to structural description.
We briefly enumerate over these tasks, since each of these individual tasks is not the focus of this report.
In this process, the synthesizer needs to decide for component selection base on the input behavior, how these components are connected to each other.
Another important task in the course of synthesis is statically cycle accurate scheduling~\cite{canis_2014_modulo,cong_2009_scheduling,cong_2008_scheduling, cong_2006_efficient}.
Synthesizing control path with either a programmable controller with a read and write program memory or just a read-only memory for fixed-functionality IPs.
And finally, model refinement is the last task that synthesizer needs to do to generate a structural description from a behavioral description.


\section{System-Level Synthesis}

A complete system contains multiple processors and shared resources that they are connected to each other with communication channels.
To synthesis a system at the behavioral level we start with a task graph to express the concurrent behavior of processes.
Figure~\ref{fig:task_graph} shows an example of behavioral description of a whole system model, this description can happen in programming languages that have support for concurrency and synchronization mechanisms.
In Figure~\ref{fig:task_graph} model, there are five processes, P1 to P5.
The system starts with P1, which conditionally triggers either P2 or another process consisting of P3, P4 and P5.
In this subprocess, P3 and P4 runs sequentially and in parallel with P5. In this figure, solid lines indicates sequential dependency, line buffers indicated concurrent relation between processes and dashed lines indicated sync points for parallel processes.
Eventually, when either P2 is finished or the sequential-parallel composition of P3, P4, and P5 is finished, the execution ends.

The behavioral model is usually a composition of two objects: processes and channels.
The structural model, on the other hand, uses different objects: processes are executed by PEs such as standard processors, custom processors and IPs, and channels are implemented by buses or network on chips with well defined protocols.
Like Processor-level synthesis at System-level synthesis, there are questions which we need to answer.
For example, how is the memory shared across different processes, what is the memory hierarchy, how to schedule the processes and so on.
Figure~\ref{fig:task_structure}, shows such system consists of standard PEs, channels, busses and memories.

\begin{figure}[!h]
    \begin{minipage}{0.5\textwidth}
       \includegraphics[width=0.85\textwidth]{figures/Introduction/Task_Graph.pdf}
       \caption{System behavioral model}
       \label{fig:task_graph}
    \end{minipage}
    \begin{minipage}{0.5\textwidth}
       \includegraphics[width=0.95\textwidth]{figures/Introduction/system_synthesis.pdf}
       \caption{System structural model}
       \label{fig:task_structure}
    \end{minipage}
    \caption{System-Level synthesis}
    \label{fig:c_example}
\end{figure}



\subsection{Missing Semantics}
\label{sec:missing_semantics}

The most challenging part of synthesis is \emph{missing semantics}.
In many cases, there are different representations and designs for a same behavioural model. 
But which model fits better in our design is the task of synthesizer to understand. 
As an example of this problem, we can look in Figure~\ref{fig:missing_semantics} at a simple \code{case statement} available in any hardware or system modelling language.
This type of case statement can be used to model a FSM in which every case expression such as X1, X2, ..., represents a state.
One more possible way of implementing this type of case statement is using loop-up table.
In the loop-up table every case X1, X2, ..., indicates a location in the memory that contains a value in the table.
Therefore,  we can use the same case statement with the same variables and format to describe two completely different components.
The challenge here is that, FSMs and look-up tables require completely different implementations:  a FSM can be implemented with a controller or with logic gates, while a look-up table is usually implemented with some kind of memory.
It is also possible to implement a FSM with a memory or a table using logic gates.
However, this implementation would not be a very efficient, and the logic's performance would not be acceptable.

The lesson is that we learn from this example is that contemporary modeling languages allow modelers to describe the design in many different ways and to use the same description for different designs details. But for automatic refinement, synthesis, we need clean and unambiguous semantic which uniquely represents all the system concepts in a given model.
Unfortunately, such a clean semantic is missing from most of the high-level synthesis tools, therefore, is very challenging for synthesizer to find the best design for a specific behavioral description.
In order to have well defined semantics, we need to introduce some form of formalism to models and modeling languages.


\begin{figure}[h]
    \centering
    \includegraphics[width=0.9\textwidth]{figures/Introduction/Missing_Semantics.pdf}
    \caption{Missing Semantics}
    \label{fig:missing_semantics}
\end{figure}


\section{Hardware Design Methodologies}

Before reviewing different languages for describing behavioral model of design, we overview design methodologies for modeling a hardware in the course of the time~\cite{michel_2012_synthesis}.

\paragraph{Method1:} Describe-and-Synthesize methodology.
In 1980s logic synthesis tools have significantly altered the design flow.
These tools could capture both the behavior and the structure of a design where both of the these information used to captured on the logic level. Before this time, there was a big gap between software and hardware design, hardware designers used to design hardware based on the software documentation and basic specification. They began the system design with block diagrams and produce gate-level design.
In the new methodology, designers specified first what they wanted in Boolean equations or FSM descriptions, and then the synthesis tools generated the implementation in terms of a logic-level netlists.
Hence, the behavior or function comes first, and the structure or implementation comes afterwards.
The main barrier of such approach for today's design is scalability, because, today’s designs are too large for this kind of methodology.
Another big change at this period of time was, the \textit{logic level} had been abstracted to the Register-Transfer Level (RTL) with the introduction of cycle-accurate modeling and synthesis.
As a result, we have two abstraction levels (RTL and logic levels) and two different models on each level (behavioral and structural).
While all these new features advanced design productivity, but the system gap still persists because there was not relation between RTL and higher system level.

\paragraph{Methode2:} Specify, Explore-and-Refine methodology.
In order to close this gap, we must increase \emph{the level of abstraction} from the RTL to the system level and to introduce a \emph{methodology} that includes both SW and HW.
The idea is that, on the system level, we can start with an executable specification that represents the system behavior; we can then \emph{extend} the system-level methodology to include several models, which they don't necessary exist at the initial executable specification, with different details that correspond to different design decisions.
Each model is used to describe some system property: functionality and application algorithms, communication, synchronization and coherence or performance and other design metrics.
We can consider each model as a specification for the next level model, in which more implementation detail is added after more design decisions are made.
Janka at ~\cite{janka_2002_specification}, named such methodology as Specify-Explore-Refine (SER), in that it consists of a sequence of models in which each model is a refinement of the previous.
Thus SER methodology follows the same designer's design process in which designers specify the intent first, then explore possibilities, and finally refine the model according to their decisions.

In the next chapter, we first overview related works for the first method, such as Hardware Definition Languages (HDL) and extensions to HLDs to increase the level of abstraction at this level. Then we look at the related works for the second methodology like High-Level Synthesis (HLS) tools and different design models involved in this process.
%!TEX root = ../main.tex

% Notes 

\chapter{Related Works}

\section{Hardware Definition Languages (HDL)}

Hardware description languages like Verilog and VHDL are designed for arbitrary circuit description.
Initially, these languages are developed as hardware simulation languages, and were only later adopted as a basis for hardware design.
While HDL language syntaxes' resemble software languages but they concepts are inherently different from software languages.
Software programs are inherently design to describe an algorithm for processors.
They have sequential semantics and correctness of the program is defined as executing instructions in the order it is written. Data movement is implicit between instructions and there is a explicit defined memory model.
In contrast, HLD languages are designed to specify circuit designs.
Hardware inherently is concurrent, therefor, each statement in HDL language is concurrent with other statements.
All the dependencies need to be explicit either in the form of wire or register.
There is no pre-defined memory model. Memory model needs to be explicitly defined and handled in the design.
Because the semantics of these languages are based around simulation, synthesizable designs must be inferred from a subset of the language, complicating tool development and designer education.
They can result in ambiguities that make automated synthesis and verification impossible, due to the unclear semantics involved.
For any design in order to achieve maximum generality, they require users to explicitly manage timing, control signals, and local memories.

A big limitation of HDL languages is that these languages lack the powerful abstraction facilities that
are common in modern software languages, which leads to low designer productivity by making it difficult to reuse components~\cite{shacham_rethinking_2010}.
Constructing efficient hardware designs requires extensive design-space exploration of alternative system microarchitectures but traditional HLDs have limited module generation facilities and are ill-suited to producing and composing the highly parameterized module generators required to support through design-space exploration.
Recent extensions such as SystemVerilog improve the type system and parameterized generate facilities but still lack many powerful programming language features.
But why reusing components is important?
The key benefit of reusing hardware components is that every time a chip is built, we inherently evaluate different design decisions, either implicitly using microarchitectural and domain knowledge, or explicitly through custom evaluation tools.
Rather than building a custom chip, designers create a template, or a chip generator, that can generate the specialized chip. Tensilica applied the same idea to create customized processors~\cite{tensillica}.

\subsection{BlueSpec}

Another alternative proposal to improve productivity of HLD languages was beginning from a domain-specific application programming language and then generate hardware blocks. 
Bluespec~\cite{bluespec} is an example of such languages.
Bluespec Compiler (BSC) is a tool that uses BluespecSystem  Verilog  (BSV)  as  the  design  language.
BSV is a high-level functional HDL based on Verilog and inspired by Haskell, where  modules are implemented as a set of rules using  Verilog  syntax.
In this language, the concurrent behavior of a system is expressed as a collection of rewrite rules.
The rules are called guarded atomic actions and express behavior in the form of concurrently cooperating finite state machines (FSMs).
Rules are predicated with a condition. They give the impression of freezing the rest of the system when a given rule's action is carried out after the rule's predicate is true.
There is an implicit parallelism in this specification, because it is possible for multiple rules to be activated and executed in parallel.
The compiler automatically schedules the rules such that they are either conflict-free or combined sequentially to preserve the promised atomicity semantic.

While these can provide great designer productivity when the task in hand matches the pattern encoded in the application programming model, they are a poor match for tasks outside their domain.
For example, the design of a programmable microprocessor is not well described in a stream
programming model, and guarded atomic actions are not a natural way to express a high-level DSP algorithm.
Furthermore, in general it is difficult to derive an efficient microarchitecture from a higher-level computation model, especially if the goal is a programmable engine to run many applications, where the human designer would prefer to write a generator to explore this design space in detail.


\section{Hardware Constructor Languages (HCL)}


\subsection{Genesis2}
Genesis2~\cite{genesis2} was one of the first attempts to work around these limitations. Genesis2 uses Perl language as a macro processing language to provide more flexible parameterization and elaboration of underlying hardware blocks in SystemVerilog. Listing~\ref{listing:genesis2} shows an example of \textit{Genesis2's code}.
In this example, perl is using ceil library from POSIX library to set the values for $\$num\_add\_bits$.

\begin{listing}[ht]
    \begin{minted}{Verilog}
        //; # More Perl Libraries
        //; use POSIX (ceil);
        //; my $reg_list = $self->define_param(REG_LIST => 
        //; [	
        //;     {name => 'regA', width => 5, default => 17, IEO => 'ie'},
        //;     {name => 'regB', width => 15, IEO => 'ieo'},
        //;     {name => 'regC', width => 32, IEO => 'ieo'},
        //; ]);
        //; my $num_regs = scalar(@{$reg_list});
        //; my $num_addr_bits = ceil(log($num_regs)/log(2));

        // Verilog code for the module
        module 'mname()' (
            input                               Clk,
            input                               Reset,
            input ['$num_addr_bits-1':0]        Addr,
            ...
            );

        endmodule // 'mname()'
    \end{minted}
    \caption[Caption for LOF]%
    {Genesis2 code example, combining SystemVerilog and Perl~\cite{genesis2}}
    \label{listing:genesis2}
\end{listing}


These approaches allow familiar and powerful languages like Verilog to be macro languages for hardware netlists, but effectively require leaf components of the design to be described in the underlying HDL.
This combined approach is cumbersome, combining the poor abstraction facilities of the underlying HDL with a completely different high-level programming model that does not understand hardware types and semantics.

\subsection{Chisel and FIRRTL}

Chisel~\cite{chisel} is another successful, modern, generalize HCL which is a domain specific language (DSL) for describing hardware circuits embedded in Scala.
Chisel provides modern programming language features such as meta-programming and object oriented concept coupled with library availability for accurately specifying low-level hardware blocks, but which can be readily extended to capture many useful high-level hardware design patterns.

There is a question that always exist about HCL languages versus HDLs. The question is that \emph{What benefits does Chisel offer over classic Hardware Description Langues?\footnote{\url{https://github.com/freechipsproject/chisel3}}}
There are two main angles: 1)Chisel improved productivity through new language features and availability of reusable libraries. 2)Improved specialization due to the hardware-compiler structure. We elaborate more on these two angles in the continue.
Chisel by itself do not provide any new hardware abstractions. However, host language features, Scala, allow designs to be more parameterizable and modular~\cite{izraelevitz_2017_firrtl_reusability}.
For instance,  someone can write a recursive Scala function to construct an adder-reduction tree, parameterized on bit-width.
Unlike the explicitly unrolled version necessary in Verilog, the same generator could be re-used anywhere an adder tree is desired.
Figure~\ref{fig:filter} shows the specified example and the abstract Chisel code.

\begin{figure}[h]
    \centering
    \includegraphics[width=0.45\textwidth]{figures/Introduction/Filter.pdf}
    \caption{Missing Semantics}
    \label{fig:filter}
\end{figure}


\begin{listing}[ht]
    \begin{minted}{Scala}

    abstract class Filter[T <: Data](dtype: T) extends Module {
    val io  = new Bundle {
    val in  = Input(Valid(dtype))
    val out = Output(Valid(dtype))
    } }

    class FunctionFilter[T <: Data](dtype: T, f: T => T) extends Filter(dtype) {
    io.out.valid := io.in.valid
    io.out.bits := f(io.in)
    }

    \end{minted}
    \caption{Chisel abstract function filter}
    \label{listing:chisel_example}
\end{listing}




In another example, designer can write a filter module which takes, as a parameter, a higher-order-function that creates the condition checking hardware.
The user of this module then only needs to write the filtering condition, re-using the base filter structure.
There are mature Chisel projects like \emph{Rocket-Chip~\cite{rocket-chip}} and \emph{Diplomacy~\cite{diplomacy}} as other examples for showing Chisel's power as hardware construction language.
The second bold advantage of Chisel compare to HDLs is a Hardware Compiler Framework that looks very much like LLVM~\cite{llvm} applied to hardware generation.
The Chisel-to-Verilog process forms part of a multi-stage compiler.
The "Chisel stage/front-end" compiles Chisel to a circuit intermediate representation called FIRRTL (Flexible Intermediate Representation for RTL)~\cite{firrtl}.
"FIRRTL stage/mid-end" then optimizes FIRRTL and applies user-custom transformations. Finally the "Verilog stage/back-end" emits Verilog based on the optimized FIRRTL.
In this process FIRRTL represents the standardized elaborated circuit that the Chisel HDL produces.
FIRRTL represents the circuit immediately after Chisel's elaboration but before any circuit simplification.
It is designed to resemble the Chisel after all meta-programming has executed. Thus, a user program that makes little use of meta-programming facilities should look almost identical to the generated FIRRT.

In fact, FIRTTL as the hos language, Chisel, to be used for mostly its met-programming facilities, the Chisel front-end can be very light-weight, and additional HDLs written in other languages can target FIRRTL and reuse the majority of the compiler toolchain.


While these improvements allow for more powerful meta-programming compared to Verilog \code{generate} statements, users still write programs at a timed circuit level. This is still one of the most important limitation of HCLs to improve the overall system design productivity.

\subsection{Spatial language}
% * Spatial~\cite{david_PLDI_2018_spatial, prabhakar_asplos_2016_parallelpattern}
% * A practice to improve chisel flexibility
% * Higher level of abstraction to chisel
% * Untimed modules
% * Reduction to parallel patterns
Recently, Spatial~\cite{david_PLDI_2018_spatial} is been proposed as a language and compiler for applications specific accelerators.
Spatial focuses on specific type of high-level abstractions required to create a new high level HDL language in which syntax contains memory, control and accelerator-host interface as an individual entity.
Spatial claims these particular constructs within the language are better fit for accelerators specially targeting applications with data locality and data parallelism.
Spatial tries to reduce productivity gap by increasing level of abstraction at the structure domain. Spatial expresses accelerators with untimed module, nested loops and customized memory hierarchy.
The most advantage of Spatial is limiting the design to these set of constructs and it allows compiler to more easily optimize designs.
Spatial in fact, is a new DSL language on top of Chisel which tries to strike balance between high-level constructs in the language for improving programmer productivity versus low-level syntax for tuning performance. 
For instance, to enable the compiler to be able to reason about the loop structures, Spatial limits the types of control structures to four types: FSMs, Foreach, Reduce and MapReduce. 
Therefore, as long as the hardware accelerator designer can express his design in such patterns, compiler can reason about the available parallelism and automatically pipeline the design.
However, in many cases such control structures are note enough to express arbitrary designs.

While Spatial has improved the accelerator design productivity compare to classic HDL designs.
But still to design a specific accelerator design needs to re-implement the accelerator's design in Spatial language. Moreover, Spatial is not designed to be target by high level synthesis tools.
Therefore, it can not support synthesizing arbitrary designs from behavioral description.
The main goal of Spatial is limiting DSE by defining higher level abstractions on top of Chisel and enabling compiler to only search trough very limited space.

\section{High-Level Synthesis (HLS)}

High-level synthesis (HLS) techniques have been proposed to improve the productivity of hardware designers by automatically generating the hardware from a high-level description of an application.
In pure C-to-gates HLS front-end which captures system behavior with a model of computation in a standard language such as C, C++, SystemC as an input, in the form of untimed nested parallel loops.
In the next step, the compiler tries to statically schedule the input algorithm and applies optimizations such as inner loop pipelining, unrolling, and memory banking and buffering~\cite{chung_micro_2010, lee_1989_new, paulin_1989_force}.
Examples of such HLS tools are LegUp~\cite{canis_2011_legup}, Vivado HLS~\cite{vivadohls}, Intel’s FPGA SDK for OpenCL~\cite{opencl_sdk}, and SDAccel~\cite{sdaccel} allow users to write FPGA designs in C/C++ and OpenCL.
Usually such tools adopt polyhedral tools to automate loop pipelining and banking decisions, but such techniques are limited to only affine accesses withing a single loop nest~\cite{wang_2014_theory}, it does not address non-affine cases or cases where accesses to the same memory occur in multiple loop nests.
For instance, Pouchet et al.~\cite{pouchet_2013_polyhedral}  explore combining HLS with polyhedral analysis to optimize input designs for locality and use estimates from HLS tools to drive design space exploration.
While this captures a larger design space than previous work by including tile sizes, this approach is limited to the capabilities of the HLS tools and to benchmarks that have strictly affine data accesses.

To elaborate more on this limitation of standard HLS approaches, consider the code in Listing~\ref{listing:static_schedule}.
In this loop there is a control flow decision (if) which depends on the actual data being read from arrays A[] and B[].
The operation which might take place in a specific iteration (s = s + d) introduces a dependency between iterations and delays the next iteration whenever the condition is true.
A typical HLS tool needs to create a static schedule for the loops, which means it needs to take the conservative decisions for synthesizing the loop. The compiler has either two choices, make the loop serial and state machine base or make the loop iterations pipeline, which for pipelining it usually need input directives from the input algorithm.
Figure~\ref{fig:schedule} shows the both possible schedule for if we use HLS tools that use statically scheduling approach.


\begin{figure*}[ht]
    \begin{minipage}{0.4\linewidth}
        \begin{minted}{C}
            float d, s = 0.0;
            for (int i = 0; i < 100; i++){
                d = A[i] - B[i];
                if (d >= 0)
                    s += d:
        }
    \end{minted}
    \end{minipage}
    \begin{minipage}{0.45\linewidth}
        \begin{minted}{C}
            float d, s = 0.0;
            #pragma pipeline
            for (int i = 0; i < 100; i++){
                d = A[i] - B[i];
                if (d >= 0)
                    s += d:
        }
    \end{minted}

    \end{minipage}



    \caption{Limitations of static scheduling}
    \label{listing:static_schedule}
\end{figure*}


\begin{figure}[!h]
    \begin{minipage}[t]{\linewidth}
        \centering
        \includegraphics[width=1\textwidth]{figures/Introduction/schedule.pdf}
        \label{fig:no_pipeline}
    \end{minipage}
    \hspace{0.1cm}
    \begin{minipage}[t]{\linewidth} 
        \centering
        \includegraphics[width=1\textwidth]{figures/Introduction/schedule_pipe.pdf}
        \label{fig:pipeline}
    \end{minipage}        
    \begin{minipage}[t]{\linewidth} 
        \centering
        \includegraphics[width=1\textwidth]{figures/Introduction/schedule_dynamic.pdf}
        \label{fig:dynamic_schedule}
    \end{minipage}        
    \label{fig:schedule}
    \caption{Static Schedule: a)no pipeline, b)pipeline}
\end{figure}  


Another example for HLS is FIR as we had the implementation in Chisel, but this time in C like language for HLS:

\begin{listing}[ht]
    \begin{minted}{C}
    #include "fir.h"

    out_data_t fir_filter(inp_data_t x, coef_t c[N]) {
        static inp_data_t shift_reg[N];
    
        acc_t acc = 0;
        acc_t mult;
        out_data_t y;
    
    Shift_Accum_Loop:
        for (int i = N - 1; i >= 0; i--) {
        #pragma HLS LOOP_TRIPCOUNT min = 1 max = 16 avg = 8
    
        if (i == 0) {
            // acc+=x*c[0];
            shift_reg[0] = x;
        } else {
            shift_reg[i] = shift_reg[i - 1];
            // acc+=shift_reg[i]*c[i];
        }
        mult = shift_reg[i] * c[i];
        acc = acc + mult;
        }
    
        y = (out_data_t)acc;
    
        return y;
    }
    \end{minted}
    \caption{FIR filter for HLS}
    \label{listing:hls_fir_filter}
\end{listing}

One solution to come across control dependencies was synthesizing hyperblocks~\cite{hyperblock}.
Through the use of predication, each hyperblock is transformed into straight-line code and then the computation portion of each hyperblock is next implemented speculatively in the form of predicated execution.
CASH~\cite{budiu_cash_2002, budiu_pegasus_2002}, AHRL~\cite{ahrl} and CHiMPS~\cite{chimps} are examples of such technique. But the main disadvantage of this paradigm is the requirement for substantial hardware resources, and it's not a scalable approach. While there are other disadvantages like limitation in memory layout~\cite{spatial_computation}.

More recent works like Elastic Circuit~\cite{elasticCircuit, elasticFlow}, CGPA~\cite{cgpa} and ~\cite{josipovic_fpga_2018_dynamically} use dynamically schedule circuit to avoid the limitations in inffering loop pipelining.
The idea is to refine from triggering the operators through centralized pre-planned controller but to take scheduling decisions locally in the circuit as it runs.
Therefore, as soon as all the conditions for execution of an operation is satisfied the has to start.
Figure~\ref{fig:dynamic_schedule} shows the execution of a dynamically scheduled HLS circuit.
The key to a good execution of this loop is that, ideally, a new value of \code{i} should be used to start computing \code{A[i] - B[i]} on every cycle.
While this techniques can dynamically provides better schedule compare to other statically scheduled HLS techniques.
But it is clear that, as in the case of processors, taking scheduling decisions dynamically costs resources and time such as the area and delay of the control elements.



Overall all the pure C-to-gates HLS techniques described up to here, rely on capturing parallelism between fine-grain operations of sequential code by constructing the control and dataflow graph (CDFG) of the computation kernels.
Then they use either scheduling algorithms or dynamic techniques to extract parallelism between the operations that are provably independent in the CDFG.
However, these techniques are less effective in capturing coarse-grain parallelism.
Hence, the Quality of Results (QoR) of these HLS tools are often lower than that of manual design process using low-level hardware-description languages.
Because the highest performance hardware exploits both fine-grain and coarse-grain parallelism.

To extract coarse-grain parallelism, main-stream HLS tools accept parallel programming constructs and/or annotations.
For example, CatapultC~\cite{catapult} and Vivado~\cite{vivado} accepts SystemC~\cite{systemc} processes and modules.
Vivado, also, accepts parallel functions/code-blocks communicating through dataflow channels.
Altera openCL compiler~\cite{opencl_sdk} accepts single instruction multiple thread (SIMT) programming model. 
CMOST~\cite{zhang_DAC_2015_cmost} is a C-to-FPGA framework that uses task-level modeling to exploit multi-level parallelism.

ParallelXL~\cite{chen_micro_2018_parallelXL} adopts continuation passing mechanism to express computation as a dynamic task graph with explicit dependencies.
Using continuation framework as foundation, ParallelXL builds other constructs such as data-parallel loops and for-join patterns on top of it.
ParallelXL implementation is a mixture of HCL languages to implement system architecture and standard HLS tools to implement task elements.

In this methodology, however, the number of input/output iterations and the execution time of each dataflow node for processing the input iteration need to be static and known at compiler time.
Due to this static nature of the synchronous data flow (SDF) model, statically scheduled HLS frameworks can explore the space of a design using closed-form formulations, resulting in micro-architectures that achieve throughput and area goals independent of the input data values.
SDF HLS, however, is restricted to regular applications such as signal processing and multi-media.
Examples of such HLS tools with regular data accesses are Rigel~\cite{hegarty_2016_rigel} and Darkroom~\cite{darkroom} which generate Verilog, and PolyMage~\cite{mullapudi_asplos_2015_polymage} that generates OpenMP and C++ for high-level synthesis.
Rigel and Darkroom support generation of specialized memory structures on FPGAs, such as line buffers, in order to capture reuse.

In irregular dataflow applications, however, the memory accesses are usually global and second the processing time of each kernels is dependent on data. 
Gorilla~\cite{lavasani_thesis}++ is a language and compiler which tackle these problems for irregular  data-stream applications.
Gorilla++ language model provides safe and efficient mechanism to access global resources using multi-threading and lock based synchronization.
To overcome on static approach's limitations, the Gorilla compiler uses profile driven optimizations to iteratively refine an accelerator using generic refinement rules for a given input data-set.
Using this approach the generated micro-architecture might not be optimized for other input data-sets however.

Lime~\cite{lime} is a Java-based programming model and runtime from IBM which aims to provide a single unified language to program heterogeneous architectures like FPGAs.
Lime only synthesis a portion of program that recognized as non-hard-to-synthesize.
Lime natively supports custom bit precisions and includes collection operations, with parallelism in such operations inferred by the compiler.
Coarse-grained pipeline and data parallelism are expressed through ``tasks''.
Coarse-grained streaming computation graphs can be constructed using built-in constructs like \texttt{\small{connect}}, \texttt{\small{split}}, and \texttt{\small{join}}.
The Lime runtime system handles buffering, partitioning, and scheduling of stream graphs.
However, coarse-grained pipelines which deviate from the streaming model are not supported, and the programmer has to use a low-level messaging API to handle coarse-grained graphs with feedback loops.
Additionally, the compiler does not perform automatic design tuning.
Finally, the compiler's ability to instantiate banked and buffered memories is unclear as details on banking multi-dimensional data structures for arbitrary access patterns are not specified.



% \subsection{Rigel}
% Rigel~\cite{hegarty_2016_rigel}
% * Generate verilog
% * Coarse-grain pipeline
% * very similar control semantics to plasticine. Use
% tokens/back pressure to allow each pipeline stage to fire at their perspective rate
% * Multi-rate line buffer template
%

% \subsection{Darkroom}
% Darkroom~\cite{darkroom}
% * target ASIC, FPGA, fast CPU
% * generate structured verilog
% * line buffer
% * scheduling for inner loop pipeline. Use ILP to improve pipeline delay
%

% \subsection{PolyMage}
% PolyMage~\cite{mullapudi_asplos_2015_polymage}
% * point=wise, stencil, sampling, operation
% * functional (in a kind of awkward way)
% * python
% * generate openMP/C++
% * generate high-level synthesis tool
% * line buffer



\paragraph{Conclusion: } While these improvements allow for more powerful meta-programming compared to Verilog \texttt{\small{generate}} statements, users still write programs at a timed circuit level.
% \subsection{Hardware generator languages}
% \begin{enumerate}
%     \item Chisel, Firrtl
%     \item Rocket core
% \end{enumerate}

%!TEX root = ../main.tex

\chapter{Summary}

In this survey, we discussed the challenges of hardware accelerator design and the barriers for designers to catch up with the growth of design complexity.
We used the Y-chart model to define different level of abstractions for designing hardware accelerators and showed to increase design productivity we need to automate the transition process between behavioral to structural domain.
We provided examples of processor and system-level synthesis and showed because of missing semantics between different domains going from behavioral description to structural description is challenging.

In the related work section, we first looked at traditional HDL languages and discussed different language extensions that try to increase the productivity of hardware design.
We argued while there are advances in design productivity with the introduction of HCLs and hardware compiler frameworks like FIRRTL but why such approaches still are aboard to increase the productivity of the whole system design.
Then we looked at different HLS tools and how they synthesis hardware from languages such as C/C++.
We discussed the different type of synthesis, different scheduling methods and elaborated more on the disadvantage of each approach.

\paragraph{Conclusion:}
Unfortunately, current HLS approaches suffer from two broad limitations that make them ill suited for studying microarchitecture tradeoffs, i) they are based on the control-driven Von-Neumann execution  model,  whereas  accelerator  architecture  typically adopt a dataflow-based execution model, and ii) they represent execution behavior and not the actual structural components of a microarchitecture is fixed and only suited for specific type of workloads.
Current HLS tools, are aware of these limitations and encourage users to scatter structural hints in the behavior description in C. However, this closely ties in behavioral correctness with microarchitecture structures, requires are challenging to modify.


%   BACK MATTER  %%%%%%%%%%%%%%%%%%%%%%%%%%%%%%%%%%%%%%%%%%%%%%%%%%%%%%%%%%%%%%
%
%   References and appendices. Appendices come after the bibliography and
%   should be in the order that they are referred to in the text.
%
%   If you include figures, etc. in an appendix, be sure to use
%
%       \caption[]{...}
%
%   to make sure they are not listed in the List of Figures.
%

\backmatter%
	\addtoToC{Bibliography}
	\bibliographystyle{plain}
	\bibliography{references}

\begin{appendices} % optional
	% \chapter{Code}
	%!TEX root = main.tex

\chapter{Code}

Bluespec counter example:

\begin{listing}[ht]
    \begin{minted}{Verilog}
    // counter + decrement from Chapter 5

    interface Counter;
        method Bit#(8) read();
        method Action load(Bit#(8) newval);
        method Action increment();
        method Action decrement();
    endinterface
    (* synthesize *)
    module mkCounter(Counter);
        Reg#(Bit#(8)) value <- mkReg(0);

        PulseWire increment_called <- mkPulseWire();
        PulseWire decrement_called <- mkPulseWire();

        rule do_increment(increment_called && !decrement_called);
            value <= value + 1;
        endrule

        rule do_decrement(!increment_called && decrement_called);
            value <= value - 1;
        endrule

        method Bit#(8) read();
            return value;
        endmethod

    
    \end{minted}
\end{listing}

\begin{listing}[ht]
    \begin{minted}{Verilog}

        method Action load(Bit#(8) newval);
            value <= newval;
        endmethod

        method Action increment();
            increment_called.send();
        endmethod

        method Action decrement();
            decrement_called.send();
        endmethod
    endmodule
    
    \end{minted}
    \caption[Caption for LOF]%
    {A counter implementation in Bluespec~\cite{bluespec}}
    \label{listing:bluespec}
\end{listing}


\end{appendices}
\end{document}

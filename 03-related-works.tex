%!TEX root = main.tex

% Notes 

\section{Hardware from High-level languages}

We discuss different languages and their level of abstraction to translates their behavioral specification to structured specification.

\section{HDLs}
While their syntax is at least reminiscent of high-level software languages, the specification of a circuit in an HDL is different from writing a software program. Software programs have a sequential execution model in which correctness is defined as the execution of each instruction or function in the order it is written.
The movement of data is implicit and is left to the underling hardware.

Memory accesses are inferred, and processors provide implicit support for interfaces to memory. By contrast, hardware designs consist of blocks of circuitry that all run concurrently. Data movement is written explicitly into the model, in the form of wires and ports. Memory and memory accesses must be explicitly declared and handled [Martinez et al., 2008]. Listing 2.1 shows the VHDL implementation of a 1-bit half adder along with its RTL implementation in Figure 2.10. A half adder adds two input bits A and B and generates two outputs a carry and sum. Table 2.1 shows the truth table of a half adder.


Hardware description languages like Verilog and VHDL are designed for arbitrary circuit description. In order to achieve maximum generality, they require users to explicitly manage timing, control signals, and local memories. Loops are expressed by state machines in flattened RTL.
One exception to this is Bluespec SystemVerilog \cite{bluespec}, which supports state machine inference from nested while loops.
Recent advancements in HDLs have largely been aimed at meta-programming improvements and increasing the size of hardware module libraries.
Languages like Chisel~\cite{chisel}, MyHDL~\cite{myhdl} and VeriScala~\cite{veriscala} make procedural generation of circuits simpler by embedding their HDL in a software language (e.g. Scala or Python). Similarly, Genesis2~\cite{genesis2} adds Perl scripting support to SystemVerilog to help drive procedural generation. While these improvements allow for more powerful meta-programming compared to Verilog \texttt{\small{generate}} statements, users still write programs at a timed circuit level.
%For application accelerators, such a low level of abstraction is often too tedious.




%
\subsection{Bluespec SystemVerilog}
Bluespec SystemVerilog~\cite{bluespec}
% * Loops are expressed by state machines
% * state machine inference from nested while loops
% 

\subsection{Spatial}
Spatial~\cite{david_PLDI_2018_spatial, prabhakar_asplos_2016_parallelpattern}


\subsection{Chisel/ MyHDL/ VeriScala / PyMTL}
Chisel~\cite{bachrach_dac_2012_chisel}/ MyHDL~\cite{decaluwe_2004_myhdl}/ VeriScala~\cite{liu_2017_scala} / PyMTL~\cite{lockhart_ISCA_2014_pymtl}
% * Embed HDL in software languages like Scala or Python
% * Adding support for meta-programming 
% * Improve modular hardware library design
%

The dominant traditional hardware-description languages
(HDLs), Verilog and VHDL, were originally developed as
hardware simulation languages, and were only later adopted
as a basis for hardware synthesis. Because the semantics of
these languages are based around simulation, synthesizable
designs must be inferred from a subset of the language, complicating tool development and designer education. These
languages also lack the powerful abstraction facilities that
are common in modern software languages, which leads to
low designer productivity by making it difficult to reuse components. Constructing efficient hardware designs requires
extensive design-space exploration of alternative system microarchitectures [9] but these traditional HDLs have limited
module generation facilities and are ill-suited to producing
and composing the highly parameterized module generators
required to support thorough design-space exploration. Recent extensions such as SystemVerilog improve the type system and parameterized generate facilities but still lack many
powerful programming language features.
To work around these limitations, one common approach
is to use another language as a macro processing language
for an underlying HDL. For example, Genesis2 uses Perl to
provide more flexible parameterization and elaboration of
hardware blocks written in SystemVerilog [9]. The language
called Verischemelog [6] provides a Scheme syntax for specifying modules in a similar format to Verilog. JHDL [1]
equates Java classes with modules. HML [7] uses standard
ML functions to wire together a circuit. These approaches
allow familiar and powerful languages to be macro languages
for hardware netlists, but effectively require leaf components
of the design to be described in the underlying HDL. This
combined approach is cumbersome, combining the poor abstraction facilities of the underlying HDL with a completely
different high-level programming model that does not understand hardware types and semantics.
An alternative approach is to begin from a domain-specific
application programming language from which a hardware
block is generated. Esterel [2] uses event-based statements
to program hardware for reactive systems. DIL [4] is an intermediate language targeted at stream processing and hardware virtualization. Bluespec [3] supports a general concurrent computation model, based on guarded atomic actions.
generator to explore this design space in detail.


In this paper, we introduce Chisel (Constructing Hardware In a Scala Embedded Language), a new hardware design language we have developed based on the Scala programming language [8]. Chisel is intended to be a simple
platform that provides modern programming language features for accurately specifying low-level hardware blocks,
but which can be readily extended to capture many useful high-level hardware design patterns. By using a flexible
platform, each module in a project can employ whichever
design pattern best fits that design, and designers can freely
combine multiple modules regardless of their programming
model. Chisel can generate fast cycle-accurate C++ simulators for a design, or generate low-level Verilog suitable for
either FPGA emulation or ASIC synthesis with standard
tools. We present several design examples and results from
emulation and synthesis experiments.
While these can provide great designer productivity when
the task in hand matches the pattern encoded in the application programming model, they are a poor match for tasks
outside their domain. For example, the design of a programmable microprocessor is not well described in a stream
programming model, and guarded atomic actions are not a
natural way to express a high-level DSP algorithm. Furthermore, in general it is difficult to derive an efficient microarchitecture from a higher-level computation model, especially
if the goal is a programmable engine to run many applications, where the human designer would prefer to write a

\subsection{Firrtl}
Firrtl~\cite{izraelevitz_2017_firrtl_reusability, li_2016_firrtl_specification}
% * Compiler



\subsection{Intorduction to HLS}
\cite{gajski_1994_introduction}

\subsection{Chimps, CASH, AHRL}

Chimps~\cite{chimps}, CASH~\cite{budiu_pegasus_2002,budiu_cash_2002}, AHRL~\cite{ahrl}
% * Generate hardware from ANSI C
% * mapping each C language construct in a dataflow graph to an HDL block.
%

\subsection{Bluespec}
Bluespec~\cite{bluespec}
% * generates hardware from purely functional description
% * Rule based hardware design language
% * The rules are atomic operations predicated with a condition.
% * They give the impression of freezing the rest of the system when a given rule’s action is carried out after the rule’s predicate is true.
% * There is an implicit parallelism in this specification, because it is possible for multiple rules to be activated and executed in parallel.
% *The compiler automatically schedules the rules such that they are either conflict-free or combined sequentially to preserve the promised atomicity semantic.
%

\subsection{Vivado/Autopilot / Legup / Intel's FPGA SDK for OpenCL, and SDAccel}

Vivado/Autopilot~\cite{vivado,vivadohls,autopilot} /Legup~\cite{canis_2011_legup} / Intel's FPGA SDK for OpenCL~\cite{opencl_sdk}, and SDAccel~\cite{sdaccel}
% * Applications can be expressed at a high level, in terms of arrays and untimed, nested loops.
% * commercial HLS tool that generates hardware from C/C++/SystemC languages
% * Imperative design description
%

\subsection{ElasticFlow / CGPA}

ElasticFlow~\cite{elasticFlow} / CGPA~\cite{cgpa} / Dynamic schedule dataflow~\cite{josipovic_fpga_2018_dynamically}
% * More recent works like ElasticFlow and CGPA
% * generate coarse-grained pipelines using FIFOs in between stages for communication.
%

\subsection{Gorill}
Gorill~\cite{lavasani_thesis}
% * Gorilla++ generates custom hardware for applications with irregular dataflow 
% * support multi-threading and lock-based synchronization

\subsection{Kiwi}
Kiwi~\cite{kiwi}
% * translates a set of C# parallel constructs to hardware units (monitor, event, lock)
Kiwi [70] translates a set of $C\#$ parallel constructs (e.g., event, monitor,
and lock) to corresponding hardware units.

\subsection{Lime}
Lime~\cite{lime}
% * hardware from functional parallel patterns. (map, reduce, split, join)
% * Java and automatically targets CPUs, GPUs, and FPGAs
%
Lime is a Java-based programming model and runtime from IBM which aims to provide a single unified language to program heterogeneous architectures. Lime natively supports custom bit precisions and includes collection operations, with parallelism in such operations inferred by the compiler. Coarse-grained pipeline and data parallelism are expressed through ``tasks''. Coarse-grained streaming computation graphs can be constructed using built-in constructs like \texttt{\small{connect}}, \texttt{\small{split}}, and \texttt{\small{join}}. The Lime runtime system handles buffering, partitioning, and scheduling of stream graphs. However, coarse-grained pipelines which deviate from the streaming model are not supported, and the programmer has to use a low-level messaging API to handle coarse-grained graphs with feedback loops. Additionally, the compiler does not perform automatic design tuning. Finally, the compiler's ability to instantiate banked and buffered memories is unclear as details on banking multi-dimensional data structures for arbitrary access patterns are not specified.

\subsection{CMOST/ParallelXL}
CMOST~\cite{zhang_DAC_2015_cmost}, ParallelXL~\cite{chen_micro_2018_parallelXL}
% * C-to-FPGA framework that uses task-level modeling



\section{Image Processing DSLs}

\subsection{HIPACC}
HIPACC~\cite{membarth_2016_hipa}
% * source - to - source C like compiler
% * generate CUDA, openCL, Renderscripts target GPU
% * make use of different memory hierarchy based on limit set of access pattern
% * support fix set of reduction function
% * unroll only if kernel size is known
% * heuristic pick best configuration (similar to our DSE)
%

\subsection{Rigel}
Rigel~\cite{hegarty_2016_rigel}
% * Generate verilog
% * Coarse-grain pipeline
% * very similar control semantics to plasticine. Use
% tokens/back pressure to allow each pipeline stage to fire at their perspective rate
% * Multi-rate line buffer template
%

\subsection{Darkroom}
Darkroom~\cite{darkroom}
% * target ASIC, FPGA, fast CPU
% * generate structured verilog
% * line buffer
% * scheduling for inner loop pipeline. Use ILP to improve pipeline delay
%

\subsection{PolyMage}
PolyMage~\cite{mullapudi_asplos_2015_polymage}
% * point=wise, stencil, sampling, operation
% * functional (in a kind of awkward way)
% * python
% * generate openMP/C++
% * generate high-level synthesis tool
% * line buffer


\subsection{Generators}

\subsection{Genesis2/ OpenPiton}
Genesis2~\cite{genesis2}
% * Adds Perl scripting support to SystemVerilog
% * Improve procedural generation
%
OpenPiton~\cite{balkind_asplos_2016_openpiton}
% * written in industry standard Verilog
% * FPGA synthesis scripts as well as ASIC backend scripts.

\paragraph{Conclusion: } While these improvements allow for more powerful meta-programming compared to Verilog \texttt{\small{generate}} statements, users still write programs at a timed circuit level.

% \subsection{Hardware generator languages}
% \begin{enumerate}
%     \item Chisel, Firrtl
%     \item Rocket core
% \end{enumerate}
